\chapter{Gramática}

\section{Pronunciación}
\begin{myitemize}
\item O: como o
\item Ö: como u
\item U: como u
\item Ü: como iu
\item A: como a
\item Ä: como e
\item SP: como schp
\item W: al principio de la palabra, como v
\item V: al principio de la palabra, como f
\item IE: como ii
\item EI: como ai
\item CH: como j
\item CH suave: luego de e, i, ei, ie, eu, äu, ö.
\item CH gutural: luego de a, o, u au
\item EU: como oi
\item ÄU: como oi
\item QU: como qv
\item CHS: como k 
\item EH: ee
\item AH: aa
\item OH: oo
\item Vocales largas: Se pronuncian un poco más de tiempo. Van siempre antes de una sola consonante,
\item Vocales cortas: Se pronuncian un tiempo corto. Van Antes de una doble consonante.
\item B al final: se pronuncia como P
\item D al final: se pronuncia como T
\item G al final: se pronuncia como K (excepto NG o IG)
\item ß: va sólo luego de vocales largas y diptongos (sólo alemán de Alemania).
\item SS: luego de vocales cortas
\end{myitemize}


\section{Sustantivos}
Todos los sustantivos comienzan con mayúscula, sin importar si están al principio o no de la palabra.
Ejemplo: Der Vater heißt Mario.

\subsection{Género}
Hay tres géneros de sustantivos: Masculino, Femenino y Neutro. Dependiendo del género, se usa un artículo en particular:
\begin{myitemize}
\item Masculino: Der
\item Femenino: Die
\item Neutro: Das
\end{myitemize}
Hay algunas reglas para poder identificar el género de una palabra. Sino la experiencia nos puede ayudar a identificarlas.
%\todo agregar luego

\subsection{Plurales}
Existen muchas maneras de formar el plural en alemán. Para cada sustantivo, hay que aprender qué regla sigue.

\begin{tabular}{|l | l l l|}
\hline
\textbf{Categoría} & Singular & Plural & Ejemplos\\
\hline
-er & Kleid & Kleider & Bild, Bilder \\
\"{ }-er & Buch & Bücher & Blatt, Blätter \\
-e & Schuh & Schuhe & Heft, Hefte \\
\"{ }-e & Rock & Röcke & Notizblock, Notizblöcke \\
-s & Handy & Handys & Laptop, Laptops \\
-n & Tasse & Tassen & Brille, Brillen \\
-en & Tür & Türen & Uhr, Uhren \\
-- & Becher & Becher & Schlüssel; Ordner \\
\hline
\end{tabular}

\section{Pronombres posesivos}
El posesivo depende del género del sustantivo. En la siguiente tabla, la primer columna identifica el locutor. El resto de las columnas identifican el género del sustantivo al que el locutor hace referencia.
\begin{tabular}{|l | l l l l|}
\hline
\textbf{Persona} & \textbf{Singular Masculino} & \textbf{Singular Femenino} & \textbf{Singular Neutro} & \textbf{Plural}\\
\hline
ich & mein & meine & mein & meine \\
du & dein & deine & dein & deine \\
er & sein & seine & sein & seine \\
sie & ihr & ihre & ihr & ihre \\
es & sein & seine & sein & seine \\
wir & unser & unsere & unser & unsere \\
ihr & euer & eure & euer & eure \\
sie/Sie & ihr/Ihr &  ihre/Ihre & ihr/Ihr &  ihre/Ihre \\
\hline
\end{tabular}

También se puede utilizar el sufijo ``s'': \textit{Das ist Ben\textbf{s} Mutter} para nombres propios (ver Genitiv), si la palabra no termina ni en \textbf{s} ni en \textbf{x}.

\section{Negación}

Tipos de negación:
\begin{myitemize}
\item nein: no (respuestas si/no). \textit{Magst du meinen Freund? Nein.}
\item nicht: verbo, adverbio o adjetivo. \textit{Mein Auto ist nicht rot.}
\item kein/e: sustantivo. \textit{Ich habe kein rotes Auto.}
\end{myitemize}
Nota: Kein se declina como ein: \textit{kein, keine, keinem, keinen,...}.

\begin{myitemize}
\item Masculino: Das ist \textbf{kein} Herd.
\item Femenino: Das ist \textbf{keine} Tür.
\item Neutro: Das ist \textbf{kein} Handy.
\item Plural: Das sind \textbf{keine} Stühle.
\end{myitemize}


\section{Partículas}
Uso de partículas
\begin{myitemize}
\item im: mes, estación. \textit{im Januar}.
\item am: día, momento del día. \textit{ am Montag}.
\item um: hora. \textit{um halb 9}.
\item von...bis: hora del día o rango de días. \textit{von 8 bis 9}.
\end{myitemize}

\section{Origen}
Para decir que uno viene de un país, se utiliza la fórmula: \textit{Ich komme aus...}. Dependiendo del país, se debe agregar un artículo del dativo: der, dem o den.

Ich komme ...
\begin{myitemize}
\item aus: Deutschland, Argentinien, Spanien, Italien, Frankreich, Japan, Belgien, Russland
\item aus der: Slowakei, Ukraine, Mongolei, Türkei, Schweiz
\item aus dem: Libanon, Irak, Iran, Kosovo, Sudan
\item aus den: USA, Niederlanden, Arabischen Emiraten
\item von den: Bahamas, Philippinen
\end{myitemize}


\section{Casos}
Los casos ayudan en alemán a organizar los elementos de los que se componen las oraciones. Cada caso indica la función que desempeña un sustantivo. La declinación hace posible que identifiquemos el caso de cada sustantivo según la terminación que adopta y la del artículo, pronombre y adjetivo que puedan acompañarlo. Además, estas terminaciones hacen posible que podamos variar el orden en el que aparecen en la oración sin alterar su significado.

\subsection{Acusativo - Akkusativ}
El acusativo se usa con algunas preposiciones y verbos para hacer referencia al objeto directo de la oración. Todas las formas son iguales al nominativo, excepto para el masculino, donde cambia la terminación. 

\begin{tabular}{| l | l | l | l |}
\hline
\textbf{Género} & \textbf{Nominativ} & \textbf{Akkusativ} & \textbf{Undefiniert} \\
\hline
Masculin & der & de\textbf{n} & ein\textbf{en}\\
Feminin & die & die & eine\\
Neutrum & das & das & ein  \\
Plural & die & die & -- \\
\hline
\end{tabular}

Preposiciones con las que se usa:
\begin{multicols}{3}
\begin{myitemize}
\item bis: hasta
\item durch: a través
\item für: para
\item gegen: frente a
\item ohne: sin
\item um: alrededor
\item wider: contra
\end{myitemize}
\end{multicols}

Verbos con los que se usa:
\begin{multicols}{3}
\begin{myitemize}
\item anrufen: llamar (teléfono)
\item besuchen: visitar
\item essen: cocinar
\item es gibt: hay...
\item fotografieren: fotografiar
\item fragen: preguntar
\item haben: tener
\item hören: escuchar
\item kaufen: comprar
\item kennen: saber
\item lesen: leer
\item mögen: querer
\item öffnen: abrir
\item sehen: ver
\item suchen: buscar
\item tragen: vestir
\item trinken: beber
\item wollen: desear
\end{myitemize}
\end{multicols}

\subsection{Dativo - Dativ}
El Dativo se asemeja al complemento indirecto del verbo, define el objeto que recibe la acción del verbo. Hay algunos verbos y preposiciones que requieren el uso del dativo.

\begin{tabular}{| l | l | l | l |}
\hline
\textbf{Género} & \textbf{Nominativ} & \textbf{Dativ} & \textbf{Undefiniert}\\
\hline
Masculin & der & dem & einem  \\
Feminin & die &  der & einer \\
Neutrum & das & dem & einem \\
Plural & die & den & -n \\
\hline
\end{tabular}

Para el plural, agregar \textbf{-n} al final, si la palabra no tiene ya una n al final.

Pronombres:

\begin{tabular}{| l | l | l |}
\hline
\textbf{Persona} & \textbf{Akkusativ} & \textbf{Dativ} \\
\hline
ich & mich & mir \\
du & dich & dir \\
er & ihn & ihm \\
sie & sie & ihr \\
es & es & ihm \\
wir & uns & uns \\
ihr & euch & euch \\
Sie/sie & Sie/sie & Ihnen/ihnen \\
\hline
\end{tabular}

Preposiciones con las que se usa:
\begin{multicols}{3}
\begin{myitemize}
\item aus: de
\item außer: excepto
\item bei: en
\item mit: con
\item nach: hacia
\item seit: desde (tiempo)
\item von: desde (lugar)
\item zu: hacia
\end{myitemize}
\end{multicols}

Verbos con los que se usa:
\begin{multicols}{3}
\begin{myitemize}
\item antworten: responder
\item befehlen: pedir
\item danken: agradecer
\item erzählen: narrar
\item fehlen: carecer
\item folgen: seguir
\item gefallen: gustar
\item gehören: pertenecer
\item glauben: creer
\item gratulieren: felicitar
\item helfen: ayudar
\item liegen: estar
\item passen: quedar
\item sagen: decir
\item schmecken: probar
\item schreiben: escribir
\item sitzen: sentar
\item stehen: estar de pie
\item vertrauen: confiar
\item zuhören: escuchar
\end{myitemize}
\end{multicols}

\subsection{Genitivo - Genitiv}
El genitivo se usa para declarar la relación de posesión entre dos complementos nominales, para decir quién es el dueño o a quién pertenece algo. Ejemplo: Das ist der Hund \textbf{meines Vaters}. Es el perro de mi padre.

El genitivo de posesión se puede remplazar con el dativo, utilizando \textbf{von}. Ejemplo: Das ist der Hund \textbf{von meinem} Vater.

Además, únicamente para nombres propios, podemos utilizar el posesivo estilo anglosajón. En este caso, se agrega la terminación \textbf{-s} al nombre. Ejemplo: Das ist Fred\textbf{s} Hund. Si el nombre ya tiene s o x al final, en lugar de \textbf{-s} se agrega únicamente un apóstrofo \textbf{Lucas` Hund}.

La declinación del genitivo se muestra a continuación: 

\begin{tabular}{| l | l | l | l |}
\hline
\textbf{Género} & \textbf{Nominativ} & \textbf{Genitiv} & \textbf{Undefiniert}\\
\hline
Masculin & der Vater & des Vaters & eines Vaters \\
Feminin & die Mutter &  der Mutter & einer Mutter \\
Neutrum & das Auto & des Autos & eines Autos\\
Plural & die Eltern & der Eltern & Eltern \\
\hline
\end{tabular}

Notar que se debe agregar \textbf{-s} a los sustantivos masculinos y neutros. Por otra parte, para palabras de una sóla sílaba, o palabras terminadas en s, ß, sch, x y z, se debe agregar \textbf{-es} al final. Ejemplo: Die Seite des Buch\textbf{es}.

Por otro lado, los sustantivos débiles (masculinos), los adjetivos sustantivados y los verbos sustantivados no utilizan la terminación \textbf{-s}, sino que utilizan las terminaciones \textbf{-n} o \textbf{-en}. 

Para preguntar de quién es algo, se utiliza \textbf{wessen}. Ejemplo: \textbf{Wessen} Eleganz will sie haben? Sie will die Eleganz ihrer Mutter haben.

El genitivo también se utiliza luego de ciertos verbos, preposiciones o adjetivos, y también para exprimir una duración.

Para exprimir una duración. Ejemplo: \textbf{Eines Morgnes} ist die Oma gefallen.

Con ciertas preposiciones:

\begin{tabular}{| l | l | l |}
\hline
\textbf{Preposición} & \textbf{Significado} & \textbf{Ejemplo}\\
\hline
anstatt & en lugar de & Er arbeitet anstatt seines Kolleguen\\
während & durante & Während der Woche wohne ich bei meiner Mutter\\
trotz & a pesar de & Trotz des Winds spielt er Tennis\\
wegen & a causa de & Wegen der Kälte werde ich krank\\
infolge & luego de & Infolge des Terroranschlags haben die Leute Angst\\
außerhalb & fuera de & Er läuft außerhalb der Stadt\\
innerhalb & dentro de & Ich bin innerhalb des Kinos\\
\hline
\end{tabular}

Luego de algunos adjetivos:

\begin{tabular}{| l | l | l |}
\hline
\textbf{adjetivo} & \textbf{Significado} & \textbf{Ejemplo}\\
\hline
sicher & seguro & Ich bin sicher der Antwort\\
bewusst & conciente & Er ist ihm bewusst des Problems\\
würdig & digno & Sie ist würdig dieses Namens\\
\hline
\end{tabular}

\subsection{Acusativo o Dativo}
Hay algunas preposiciones que se pueden usar con ambos casos:
\begin{multicols}{3}
\begin{myitemize}
\item an: contra
\item auf: sobre
\item hinter: detrás
\item in: en
\item neben: al lado de
\item über: sobre
\item unter: bajo
\item vor: delante
\item zwischen: entre
\end{myitemize}
\end{multicols}

Si la preposición responde a la pregunta \textit{dónde} (wo), se usa el dativo. \textit{Wo bleibst du? In die Schule}. 

Se utiliza también para todos los verbos de posición:
\begin{multicols}{3}
\begin{myitemize}
\item bleiben: permanecer
\item hängen: estar colgado
\item liegen: estar acostado
\item schlafen: dormir
\item sein: estar
\item sitzen: estar sentado
\item stehen: estar de pie
\item warten: esperar
\item wohnen: vivir
\end{myitemize}
\end{multicols}

Si la preposición responde a la pregunta \textit{hacia dónde} (wohin), se usa el acusativo. \textit{Wohin gehst du? In die Schule.} 

Se usa también para los verbos de movimiento:
\begin{multicols}{3}
\begin{myitemize}
\item fahren: conducir
\item fliegen: volar
\item gehen: ir
\item hängen: colgar
\item kommen: venir
\item legen: acostarse
\item setzen: sentarse
\item stellen: posar
\end{myitemize}
\end{multicols}

\subsection{Otras declinaciones}
\begin{tabular}{| l | l | l | l | l |}
\hline
\textbf{Género} & \textbf{Nominativ} & \textbf{Akkusativ} & \textbf{Dativ} & \textbf{Genitiv}\\
\hline
Maskulin & welcher & welchen & welchem & welches \\
Feminin  & welche  & welche  & welcher & welcher \\
Neutral  & welches & welches & welchem & welches \\
Plural   & welche  & welche  & welchen & welcher \\
\hline
\end{tabular}

\begin{tabular}{| l | l | l | l | l |}
\hline
\textbf{Género} & \textbf{Nominativ} & \textbf{Akkusativ} & \textbf{Dativ} & \textbf{Genitiv}\\
\hline
Maskulin & dieser & diesen & diesem & dieses \\
Feminin  & diese  & diese  & dieser & dieser \\
Neutral  & dieses & dieses & diesem & dieses \\
Plural   & diese  & diese  & diesen & dieser \\
\hline
\end{tabular}

\section{Orden de palabras en una oración}
Este es el orden preferido de palabras dentro de una oración o proposición:
\begin{myitemize}
\item S: sujeto
\item V: verbo
\item D: dativo
\item T: modificador temporal
\item C: modificador causal
\item M: modificador modal
\item A: acusativo
\item L: modificador local
\item V2: auxiliar del verbo
\end{myitemize}

Pero como los casos ayudan a identificar la función que cumple cada parte de la oración, el orden puede ser cambiado casi a voluntad.

\section{Números}
\subsection{Números ordinales}
Los números ordinales son utilizados para indicar un orden. También se usan para las fechas: am ersten Januar = 1. Januar = Primero de Enero. Para los números entre el 1 y el 19, se usa el sufijo \textbf{-ten}. Para los números mayores o iguales a 20, se usa el sufijo \textbf{-sten}. Hay también algunas excepciones en la forma de escribir.
\begin{multicols}{2}
\begin{myitemize}
\item 1: ersten
\item 2: zweiten
\item 3: dritten
\item 4: vierten
\item 5: fünften
\item 6: sechsten
\item 7: siebten
\item 8: achten
\item 9: neunten
\item 10: zehnten
\item 11: elften
\item ...
\item 20: zwanzigsten
\item 21: einundzwanzigsten
\item 22: zweiwundzwanzigsten
\item ...
\end{myitemize}
\end{multicols}

\subsection{Fracciones}
\begin{multicols}{2}
\begin{myitemize}
\item 1/2: ein halb
\item 1/3: ein drittel
\item 1/4: ein viertel
\item 1/20: ein zwanzigstel
\end{myitemize}
\end{multicols}
Nota: la palabra halb cumple la función de adjetivo y por lo tanto debe declinarse: \textit{ein halbe Liter Wasser}.


\section{Preposiciones y contracciones}
\begin{tabular}{| l | l | l |}
\hline
\textbf{Preposición} & \textbf{Partícula} & \textbf{Resultado} \\
\hline
an & das & ans \\
an & dem & am \\
auf & das & aufs \\
bei & dem & beim \\
in & das & ins \\
in & dem & im \\
zu & dem & zum \\
zu & der & zur \\
\hline
\end{tabular}

\section{Declinación de adjetivos}
Cuando el adjetivo es atributivo y va \textbf{después} del sustantivo, no debe declinarse: Das Kind ist schön.

Los adjetivos deben declinarse cuando acompañan a un sustantivo y lo califican (adjetivos calificativos, epítetos). La declinación normalmente agrega `-e` al final del adjetivo. En este caso, el adjetivo va \textbf{antes} del sustantivo: Das schöne Kind. 

Hay dos escenarios distintos. El primero es cuando consideramos que el artículo que acompaña al sustantivo ``cumple su función'' de definir bien el caso del sustantivo.
En el segundo escenario, el artículo no ``cumple su función'' (\textbf{ein}, \textbf{kein}, \textbf{mein}), o directamente no hay artículo. En este caso, el adjetivo debe definir el caso del sustantivo. Aquí, las terminaciones a agregar pueden ser \textbf{-e}, \textbf{-er} o \textbf{-es}.

Ejemplos:
\begin{myitemize}
\item Der klug\textbf{e} Mann (nominativ)
\item Die klug\textbf{e} Frau (nominativ)
\item Das klug\textbf{e} Kind (nominativ)
\item den klug\textbf{en} Mann (akkusativ)
\item dem klug\textbf{en} Mann (dativ)
\item der klug\textbf{e} Frau (dativ)
\item dem klug\textbf{en} Kind (dativ)
\item Klug\textbf{er} Mann braucht klug\textbf{es} Auto (no hay artículo)
\item Ein klug\textbf{er} Mann braucht ein klug\textbf{es} Auto (artículo defectuoso)
\item Ihr klug\textbf{er} Mann braucht ihr klug\textbf{es} Auto
\item Das klug\textbf{e} Auto gehört dem klug\textbf{en} Mann
\end{myitemize}

Las tablas completas de declinación son las siguientes.


\begin{tabular}{| l | l | l | l | l |}
\hline
\textbf{Caso} & \textbf{Masculino} & \textbf{Neutro} & \textbf{Femenino} & \textbf{Plural} \\
\hline
\textbf{Nominativ} & der nette Mann & das nette Kind & die nette Frau & die netten Kinder \\
\textbf{Akkusativ} & den netten Mann & das nette Kind & die nette Frau & die netten Kinder \\
\textbf{Dativ} & dem netten Mann & dem netten Kind & der netten Frau & den netten Kindern \\
\textbf{Genitiv} & das netten Manns & des netten Kinds & der netten Frau & der netten Kinder \\
\hline
\end{tabular}



\begin{tabular}{| l | l | l | l | l |}
\hline
\textbf{Caso} & \textbf{Masculino} & \textbf{Neutro} & \textbf{Femenino} & \textbf{Plural} \\
\hline
\textbf{Nominativ} & ein netter Mann & ein nettes Kind & eine nette Frau &  nette Kinder \\
\textbf{Akkusativ} & einen netten Mann & ein nettes Kind & eine nette Frau & nette Kinder \\
\textbf{Dativ} & einem netten Mann & einem netten Kind & einer netten Frau & netten Kindern \\
\textbf{Genitiv} & eines netten Manns & eines netten Kinds & einer netten Frau & netter Kinder \\
\hline
\end{tabular}



\begin{tabular}{| l | l | l | l | l |}
\hline
\textbf{Caso} & \textbf{Masculino} & \textbf{Neutro} & \textbf{Femenino} & \textbf{Plural} \\
\hline
\textbf{Nominativ} & netter Mann & nettes Kind & nette Frau &  nette Kinder \\
\textbf{Akkusativ} & netten Mann & nettes Kind & nette Frau & nette Kinder \\
\textbf{Dativ} & nettem Mann & nettem Kind & netter Frau & netten Kindern \\
\textbf{Genitiv} & netten Manns & netten Kinds & netter Frau & netter Kinder \\
\hline
\end{tabular}

\section{Comparativos y Superlativos}
\begin{tabular}{| l | l | l | l | l |}
\hline
\textbf{Adjetivo} & \textbf{Regla} & \textbf{Ejemplo} & \textbf{Komparativ} & \textbf{Superlativ} \\
\hline
General & -er, -sten & klein & kleiner & am kleinsten \\
-t, -d, -s, -z & -er, -esten & laut & lauter & am lautesten \\
o - ö, a - ä, u - ü & \"{ }-er, \"{ }-sten & groß & größer & am größten \\
Irregular & - & teuer & teurer & am teuersten \\
Irregular & - & viel & mehr & am meisten \\
Irregular & - & gern & lieber & am liebsten \\
Irregular & - & gut & besser & am besten \\
\hline
\end{tabular}

Ejemplos: 
\begin{myitemize}
\item Der Zug ist \textbf{schnellner} als das Fahrrad. Das Flugzeug ist \textbf{am schnellsten}.
\item Dr. House ist \textbf{der beste} Artz.
\end{myitemize}

\subsection{Igualdad}
Se usa la forma \textit{so + adj + wie} o \textit{genauso + adj + wie}: Er ist \textbf{so groß wie} seine Mutter.

\subsection{Inferioridad}
Se usa la forma \textit{nicht so + adj + wie}: Der Hund rennt \textbf{nicht so schnell wie} sein Herrchen.

\subsection{Superlativo}
El superlativo puede formarse a partir de adverbio o del adjetivo calificativo.
\subsubsection{Superlativo de adverbios}
Se forman utilizando la forma \textbf{am + adv + -sten}: Das Pferd Thunder rannte \textbf{am schnellsten}.

\subsubsection{Superlativos de adjetivos calificativos}
Se forman agregando \textbf{-st} al adjetivo antes de agregar la marca de declinación: Ich habe das \textbf{schnellste} Auto der Welt.

\section{Adjetivos sustantivados}
Los adjetivos sustantivados son sustantivos formados a partir de un adjetivo: gut $\rightarrow$ das Gute.

Siempre van acompañados de un artículo y siguen las reglas de declinación de los adjetivos. En la mayoría de los casos, se declinan añadiendo la terminación \textbf{-en}.

\begin{tabular}{| l | l | l | l | l |}
\hline
\textbf{Género} & \textbf{Nominativ} & \textbf{Akkusativ} & \textbf{Dativ} & \textbf{Genitiv}\\
\hline
Maskulin & der Neue  & den Neuen & dem Neuen & des Neuen \\
Feminin  & die Neue  & die Neue  & der Neuen & der Neuen \\
Neutral  & das Neue  & das Neue  & dem Neuen & des Neuen \\
Plural   & die Neuen & die Neuen & den Neuen & der Neuen \\
\hline
\end{tabular}

\begin{tabular}{| l | l | l | l | l |}
\hline
\textbf{Género} & \textbf{Nominativ} & \textbf{Akkusativ} & \textbf{Dativ} & \textbf{Genitiv}\\
\hline
Maskulin & ein Neuer  &  einen Neuen & einem Neuen & eines Neues \\
Feminin  & eine Neue  &  eine Neue   & einer Neuen & einer Neuen \\
Neutral  & ein Neues  &  ein Neues   & einem Neuen & eines Neuen \\
Plural   & eine Neuen &  einen Neuen & einen Neuen & einer Neuen \\
\hline
\end{tabular}

\section{N-deklination}

Algunos nombres masculinos (masculinos débiles) deben ser declinados con esta regla al agregarles \textbf{-n} o \textbf{-en}.
\begin{myitemize}
\item \textbf{-e}: der Junge, der Kunde
\item \textbf{-ent}: der Student
\item \textbf{-ant}: der Praktikant
\item \textbf{-ist}: der Journalist
\item Procendentes del griego: der Soldat, der Architekt
\item Otros: Mensch, Nachbar, Herr
\end{myitemize}

\begin{tabular}{| l | l | l | l |}
\hline
\textbf{Nominativ} & \textbf{Akkusativ} & \textbf{Dativ} & \textbf{Genitiv}\\
\hline
der Name & den Namen & dem Namen & des Namens \\
der Journalist & den Journalisten & dem Journalisten  & des Journalisten \\
der Dozent & den Dozenten & dem Dozenten & des Dozenten \\
\hline
\end{tabular}

\section{Proposiciones adverbiales - Präpositionaladverbien}
Son palabras que se usan en frases con verbos con preposiciones. También tenemos palabras interrogativas relacionadsa

Verbos con preposiciones y sus palabras interrogativas relacionadas (al referirse a \textbf{cosas}):

\begin{tabular}{| l | l | l |}
\hline
\textbf{Verb mit Prä.} & \textbf{Fragewort: wo + Prä.} & \textbf{Präpositionaladverb: da + Prä.}\\
\hline
träumen von & wovon & davon \\
warten auf & worauf & darauf \\
sprechen über & worüber & darüber\\
\hline
\end{tabular}

Otros verbos: sich beschweren bei, sich freuen auf, sich interessieren für, sich erinnen an.

Nota: si la preposición empieza con una vocal, se debe agregar la letra `R` entre \textit{wo/da} y la preposición. 

Para utilizar los adverbios davon, darauf, etc, necesitamos un contexto: algo que se mencionó antes. 

Por otro lado, si nos referimos a personas, no se utilizan las formas anteriores, sino simplemente: von wen, auf wen, von ihn, auf ihn, etc.

Ejemplos:

\textbf{Wovon} träumst du? Ich träume \textbf{ von einem Haus}. \textbf{Davon} träume ich auch.

\textbf{Worauf} wartest du? Ich warte \textbf{auf mein Resultat}. \textbf{Darauf} warte ich auch.

\textbf{Von wem} träumst du? Ich träume \textbf{von Brad Pitt}. \textbf{Von dem} träume ich gar nicht.

\textbf{Auf wen} wartest du? Ich warte \textbf{auf den Lehrer}. Ich warte auch \textbf{auf ihn}.

Ich habe \textbf{die Prufung} nicht bestanden. \textbf{Damit} habe ich nicht gerechnet.

\subsection{Doch}
\textbf{Doch} tiene varios usos.

Para contrarrestar un negativo:
Hast du kein Geld? Doch!: No tienes dinero? Sí, si tengo.

Para decir ``finalmente'' o ``en realidad'':
\begin{myitemize}
\item Ich gehe doch nicht ins Kino. Finalmente no voy al cine.
\item Der Film gefällt mir doch. En realidad sí me gustó la película.
\end{myitemize}

Para decir ``pero'':
\begin{myitemize}
\item Ich bin müde, doch ich muss diesen Post zuende lesen. Estoy cansado, pero debo terminar de leer este post.
\end{myitemize}


\section{Oraciones complejas}
\subsection{Cláusula subordinada - Nebensatz}
Una oración compleja (Satzgefüge) está formada por una cláusula principal (Hauptsatz) y una cláusula subordinada (Nebensatz). Para formarla, se usan conjuciones, por ejemplo \textbf{weil} después de la coma, seguida de la justificación, pero con el verbo conjugado al final.

Ejemplos: 
\begin{myitemize}
\item \textbf{Weil/Porque}: Ich backe einen Kuchen, \textbf{weil} mein Freund Geburstag \textit{hat}. Yo cocino un pastel \textbf{porque} mi amigo \textit{está} de cumpleaños.
\item \textbf{Da/Porque}: Für Sie ein Ausbildungsberuf besser als ein Studium, \textbf{da} Sie nicht gern am Schreibtisch sitzen. Una formación es mejor para ti que un programa de grado \textbf{porque} no te gusta estar sentado en un escritorio.
\item \textbf{Dass/Que}: Mein Bruder sagt, \textbf{dass} das Buch zu teuer \textit{ist}. Mi hermano dice \textbf{que} el libro \textit{es} muy caro.
\item \textbf{Ob/Si}: Ich frage deinen Freund, \textbf{ob} er einen Garten \textit{hat}. Le pregunto a tu amigo \textbf{si} él \textit{tiene} un jardín.
\item \textbf{Wenn/Cuando}: Ich esse einen Hamburger, \textbf{wenn} ich einen Kater \textit{habe}. Como una hamburguesa \textbf{cuando} \textit{tengo} una resaca.
\item \textbf{Falls/Cuando}: \textbf{Falls} Sie das Essen bereits beendet haben, \textit{legen} Sie die Serviette neben den Teller. \textbf{Cuando} halla terminado de comer, \textit{coloque} la servilleta junto al plato.
\item \textbf{Obwohl/Aunque}: Es hat sich bis Heute nichts geändert, \textbf{obwohl} ich Ihnen des mehrfach erklärt \textit{habe}: Nada ha cambiado a día de hoy, \textbf{aunque} te lo \textit{he} explicado varias veces .
\item \textbf{Bevor/Antes de}: Ich frühstücke, \textbf{bevor} ich zur Arbeit \textit{fähre}. Desayuno \textbf{antes de} \textit{ir} al trabajo
\item \textbf{Während/Mientras}: Ich frühstücke, \textbf{während} ich zur Arbeit \textit{fähre}. Desayuno \textbf{mientras} \textit{voy} al trabajo.
\item \textbf{Trotz/A pesar de}: \textbf{Trotz} des schlechten Wetters gingen wir spazieren. \textbf{A pesar del} mal tiempo fuimos a pasear.
\end{myitemize}

Con algunas conjunciones, la cláusula subordinada también puede ir al inicio de la oración:
\begin{myitemize}
\item obwohl: Obwohl es stark regnet, gehen die Kinder in den Park: Aunque llueve mucho, los niños van al parque.
\end{myitemize}

\subsubsection{Oraciones con Nachdem y Bevor}
Ambas preposiciones sirven para conectar una Hauptsatz con una Nebenstazt, donde una acción ocurre antes que otra. 

\textbf{Bevor} se utiliza para introducir la acción que ocurre en segundo lugar, utilizando el Präteritum. La oración subordinada utiliza el Plusquamperfekt e indica qué se hizo \textbf{antes} de la otra acción. Como se trata de dos acciones en el pasado, se deben diferenciar con los tiempos verbales utilizados: Plusquamperfekt o Präteritum.

\textbf{Nachdem} se utiliza para introducir la primer acción, y por lo tanto utiliza el Plusquamperfekt. La otra acción, que ocurre \textbf{después}, se escribe con el Präteritum.

Las estructuras son:

\textbf{Bevor} + \textit{Präteritum}, \textit{Plusquamperfekt}.

\textbf{Nachdem} + \textit{Plusquamperfekt}, \textit{Präteritum}.

Ejemplos:

\textbf{Bevor} ich den Tee \textit{trank}, \textit{hatte} ich ihn \textit{gekocht}. \textbf{Antes} de beber el té, lo había preparado.

\textbf{Nachdem} ich den Tee \textit{gekocht} \textit{hatte}, trank ich ihn. \textbf{Después} de preparar el té, me lo bebí.


\subsection{Dos cláusulas principales}
En este caso, se trata de dos cláusulas principales, conectadas por algún conector. Como se trata de cláusulas principales, el verbo no va al final sino al principio, pero hay una inversión.

Ejemplos:
\begin{myitemize}
\item trotzdem: Es regnet sehr stark, \textbf{trotzdem} \textit{wollen} die Kinder im Park spielen: Llueve muy fuerte, sin embargo los niños quieren. jugar en el parque.
\end{myitemize}


\subsubsection{Oraciones con Davor}
La preposición \textbf{davor} sirve para enlazar dos oraciones principales e indicar que una acción ocurrió antes de la otra. Como hay dos acciones en el pasado, una se debe especificar con el \textit{Perfekt} (la que ocurre segundo), y la otra con el \textit{Plusquamperkeft} (la que ocurre primero, marcada con \textbf{davor}):

\textit{Perfekt}. \textbf{Davor} + \textit{Plusquamperfekt}.

Ejemplos:

Eine Freundin hat mich zum Essen eingeladen. \textbf{Davor} hatte ich aber schon zu Mittag gegessen. Una amiga me invitó a almorzar. Pero yo \textbf{previamente} ya habia almorzado

Ich \textit{bin} um sieben Uhr \textit{aufgestanden}. \textbf{Davor} \textit{hatte} ich meinen Instagram Account noch \textit{gecheckt}. Me levanté a las siete. \textbf{Antes de eso}, revisé mi cuenta de Instagram.



\subsection{Cláusulas relativas - Relativsätze}
Las cláusulas relativas sirven para hacer referencia a algo que se mencionó antes sin repetirlo. Para formarlas, se debe separar las cláusulas con una coma, poner la preposición (si necesario), el pronombre (ver tabla) que hace referencia al elemento repetido, y ubicar el verbo al final. El pronombre a utilizar depende de la función que tiene el elemento repetido en la cláusula principal.

Ejemplo: Das sind die Freunde. Ich verbringe mit den Freunden viel Zeit. 

En este caso, \textit{Freunde} está repetido y está en dativo. La oración reformulada con cláusula relativa es: Das sind die \textit{Freunde}, \textbf{mit} \textit{denen} ich viel Zeit verbringe.

Otro ejemplo en el cual se ve que la cláusula relativa está en el medio de la oración: Viele \textit{Leute}, \textit{die} auf dem Zug warteten, standen am Bahnsteig.


La siguiente tabla muestra los pronombres a utilizar como conectores, según el caso.

\begin{tabular}{| l | l | l | l | l |}
\hline
\textbf{Género} & \textbf{Nominativ} & \textbf{Akkusativ} & \textbf{Dativ} & \textbf{Genitiv} \\
\hline
Maskulin & der/welcher & den/welchen & dem/welchem & dessen \\
Feminin  & die/welche  & die/welche  & der/welcher & deren \\
Neutral  & das/welches & das/welches & dem/welchem & dessen \\ 
Plural   & die/welche  & die/welche  & \textbf{denen}/welchen & deren \\
\hline
\end{tabular}


\subsection{Cláusulas con infinitivo - Infinitivsätze}
Se componen de dos cláusulas, la principal como siempre, y la subordinada con un verbo en infinitivo. La cláusula subordinada va separada por una coma. Como el sujeto de la cláusula subordinada es el mismo que en la principal, no se pone el sujeto en la segunda cláusula. El verbo va en infinitivo, precedido por \textbf{zu}. Si el verbo tiene partículas separables, el \textbf{zu} se ubica entre la partícula y el verbo: ab\textbf{zu}waschen.


Se utilizan con estructuras bien particulares, listadas aquí abajo.

\begin{myitemize}
\item verbos específicos: anfangen, vergessen, vorhaben, empfahlen, hoffen, aufhören, raten. Ich fänge bald an, Japanish \textbf{zu lernen}.
\item Namen + haben: Zeit haben, Lust haben, Angst haben, Problem haben. Ich habe keine Zeit, das Geschirr \textbf{abzuwaschen}.
\item Constructiones con \textbf{es}: es ist toll, es macht Spaß, es ist gesund. Es ist gesund, täglich Sport \textbf{zu machen}.
\end{myitemize}

Nota: si la claúsula subordinada se compone únicamente del verbo, se puede remover la coma: Ich habe lust \textbf{zu essen}.

\subsection{Cláusulas condicionales - Konditionalsatz}
Se utilizan para exprimir un deseo sobre una situación actual. Hay dos maneras de formar estas frases, ya sea utilizando la partícula \textbf{wenn}, o sin usarla. Además, se usa el verbo en Konjunktiv II en Präteritum.

Ejemplos:

\textbf{Wenn} ich nur mehr Zeit \textbf{hätte}, \textbf{könnte} ich mehr Sport machen.

\textbf{Hätte} ich nur mehr Zeit, \textbf{könnte} ich mehr Sport machen.

En ambos casos, la tradución es: Si \textbf{tuviera} más tiempo, \textbf{podría} hacer más deporte. 

La segunda parte del ejemplo se puede omitir. En este caso, se llama un Konditional partial.

Para reforzar la idea, podemos agregar Modelpartikeln: nur, doch y bloß.
Estas palabras van siempre en la tercer posición, excepto si el verbo es reflexivo. En este último caso, van en la cuarta posición. En general, van lo más antes posible.

Sus significados son más o menos iguales y a veces son hsata intercambiables. 
\begin{myitemize}
\item nur: solamente, sólo
\item doch: aún así, no obstante
\item bloß: tan sólo, simplemente
\end{myitemize}

Ejemplos:

\textbf{Hätte} ich \textbf{doch} früher angerufen! Ojalá hubiera llamado antes!

\textbf{Hätte} ich früher angerufen! Si hubiera llamado antes!

\textbf{Wäre} sie \textbf{nur} single! Ojalá estuviera soltera!

\textbf{Hätte} ich mir \textbf{bloß} Bitcoins gekauft! Ojalá hubiera comprado bitcoins!

\subsubsection{Diferentes tipos de irrealidades}
\textbf{Futuro posible}: Wenn ich Zeit habe, dann werde ich mit dir ins Kino gehen. Si tengo tiempo, iré al cine contigo.
\textbf{Presente irreal}: Wenn ich Zeit hätte, dann würde ich mit dir ins Kino gehen. Si tuviera tiempo, iría contigo al cine.
\textbf{Pasado irreal}: Wenn ich gestern Zeit gehabt hätte, wäre ich mit dir ins Kino gegangen. Si ayer hubiera tenido tiempo, habría ido contigo al cine.