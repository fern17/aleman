\chapter{Gramática}

\section{Pronunciación}
\begin{myitemize}
\item O: como o
\item Ö: como u
\item U: como u
\item Ü: como iu
\item A: como a
\item Ä: como e
\item SP: como schp
\item W: al principio de la palabra, como v
\item V: al principio de la palabra, como f
\item IE: como ii
\item EI: como ai
\item CH: como j
\item CH suave: luego de e, i, ei, ie, eu, äu, ö.
\item CH gutural: luego de a, o, u au
\item EU: como oi
\item ÄU: como oi
\item QU: como qv
\item CHS: como k 
\item EH: ee
\item AH: aa
\item OH: oo
\item Vocales largas: Se pronuncian un poco más de tiempo. Van siempre antes de una sola consonante,
\item Vocales cortas: Se pronuncian un tiempo corto. Van Antes de una doble consonante.
\item B al final: se pronuncia como P
\item D al final: se pronuncia como T
\item G al final: se pronuncia como K (excepto NG o IG)
\item ß: va sólo luego de vocales largas y diptongos (sólo alemán de Alemania).
\item SS: luego de vocales cortas
\end{myitemize}


\section{Sustantivos}
Todos los sustantivos comienzan con mayúscula, sin importar si están al principio o no de la palabra.
Ejemplo: Der Vater heißt Mario.

\subsection{Género}
Hay tres géneros de sustantivos: Masculino, Femenino y Neutro. Dependiendo del género, se usa un artículo en particular:
\begin{myitemize}
\item Masculino: Der
\item Femenino: Die
\item Neutro: Das
\end{myitemize}
Hay algunas reglas para poder identificar el género de una palabra, pero si no la experiencia nos puede ayudar a identificarlas.
%\todo agregar luego

\subsection{Plurales}
Existen muchas maneras de formar el plural en alemán. Para cada sustantivo, hay que aprender qué regla sigue.

\begin{tabular}{|c | c c c|}
\hline
\textbf{Categoría} & Singular & Plural & Ejemplos\\
\hline
-er & Kleid & Kleider & Bild, Bilder \\
\"{ }-er & Buch & Bücher & Blatt, Blätter \\
-e & Schuh & Schuhe & Heft, Hefte \\
\"{ }-e & Rock & Röcke & Notizblock, Notizblöcke \\
-s & Handy & Handys & Laptop, Laptops \\
-n & Tasse & Tassen & Brille, Brillen \\
-en & Tür & Türen & Uhr, Uhren \\
-- & Becher & Becher & Schlüssel; Ordner \\
\hline
\end{tabular}

\section{Pronombres posesivos}
El posesivo depende del género del sustantivo. En la siguiente tabla, la primer columna identifica el locutor y el resto de las columnas identifican el género del sustantivo al que el locutor hace referencia.
\begin{tabular}{|c | c c c c|}
\hline
\textbf{Persona} & \textbf{Singular Masculino} & \textbf{Singular Femenino} & \textbf{Singular Neutro} & \textbf{Plural}\\
\hline
ich & mein & meine & mein & meine \\
du & dein & deine & dein & deine \\
er & sein & seine & sein & seine \\
sie & ihr & ihre & ihr & ihre \\
es & sein & seine & sein & seine \\
wir & unser & unsere & unser & unsere \\
ihr & euer & eure & euer & eure \\
sie/Sie & ihr/Ihr &  ihre/Ihre & ihr/Ihr &  ihre/Ihre \\
\hline
\end{tabular}

También se puede utilizar el sufijo \" s \": Das ist Ben\textbf{s} Mutter, solamente si ya la palabra no termina en s ni en x.

\section{Negación}

Tipos de negación:
\begin{myitemize}
\item nein: no (respuestas si/no). \textit{Magst du meinen Freund? Nein.}
\item nicht: verbo, adverbio o adjetivo. \textit{Mein Auto ist nicht rot.}
\item kein/e: sustantivo. \textit{Ich habe kein rotes Auto.}
\end{myitemize}
Nota: Kein se declina como ein: \textit{kein, keine, keinem, keinen,...}.

\begin{myitemize}
\item Masculino: Das ist kein Herd.
\item Femenino: Das ist keine Tür.
\item Neutro: Das ist kein Handy.
\item Plural: Das sind keine Stühle.
\end{myitemize}


\section{Partículas}
Uso de partículas
\begin{myitemize}
\item im: mes, estación. \textit{im Januar}.
\item am: día, momento del día. \textit{ am Montag}.
\item um: hora. \textit{um halb 9}.
\item von...bis: hora del día o rango de días. \textit{von 8 bis 9}.
\end{myitemize}

\section{Origen}
Para decir que uno viene de uno país, se utiliza la fórmula: \textit{Ich komme aus...}. Dependiendo del país, se debe agregar un artículo del dativo: der, dem o den.

Ich komme ...
\begin{myitemize}
\item aus: Deutschland, Argentinien, Spanien, Italien, Frankreich, Japan, Belgien, Russland
\item aus der: Slowakei, Ukraine, Mongolei, Türkei, Schweiz
\item aus dem: Libanon, Irak, Iran, Kosovo, Sudan
\item aus den: USA, Niederlanden, Arabischen Emiraten
\item von den: Bahamas, Philippinen
\end{myitemize}


\section{Casos}
Los casos ayudan en alemán a organizar los elementos de los que se componen las oraciones. Cada caso indica la función que desempeña un sustantivo. La declinación hace posible que identifiquemos el caso de cada sustantivo según la terminación que adopta y la del artículo, pronombre y adjetivo que puedan acompañarlo. Además, estas terminaciones hacen posible que podamos variar el orden en el que aparecen en la oración sin alterar su significado.

\subsection{Acusativo - Akkusativ}
El acusativo se usa con algunas preposiciones y verbos para hacer referencia al objeto directo de la oración. Todas las formas son iguales al nominativo, excepto para el masculino, donde cambia la terminación. 

\begin{tabular}{|c | c | c |}
\hline
\textbf{Género} & \textbf{Nominativ} & \textbf{Akkusativ} \\
\hline
Masculin & der Mann & de\textbf{n} Mann \\
Masculin & ein Mann & ein\textbf{en} Mann \\
Feminin & die Frau & die Frau \\
Feminin & eine Frau & eine Frau \\
Neutrum & das Kind & das Kind \\
Neutrum & ein Kind & ein Kind \\
Plural & die Leute & die Leute\\
\hline
\end{tabular}

Preposiciones con las que se usa:
\begin{multicols}{3}
\begin{myitemize}
\item bis: hasta
\item durch: a través
\item für: para
\item gegen: frente a
\item ohne: sin
\item um: alrededor
\item wider: contra
\end{myitemize}
\end{multicols}

Verbos con los que se usa:
\begin{multicols}{2}
\begin{myitemize}
\item anrufen: llamar (teléfono)
\item besuchen: visitar
\item essen: cocinar
\item es gibt: hay...
\item fotografieren: fotografiar
\item fragen: preguntar
\item haben: tener
\item hören: escuchar
\item kaufen: comprar
\item kennen: saber
\item lesen: leer
\item mögen: querer
\item öffnen: abrir
\item sehen: ver
\item suchen: buscar
\item tragen: vestir
\item trinken: beber
\item wollen: desear
\end{myitemize}
\end{multicols}

\subsection{Dativo - Dativ}
El Dativo se asemeja al complemento indirecto del verbo, define el objeto que recibe la acción del verbo. Hay algunos verbos y preposiciones que requieren el uso del dativo.

\begin{tabular}{|c | c | c | c |}
\hline
\textbf{Género} & \textbf{Nominativ} & \textbf{Dativ} & \textbf{Undefiniert}\\
\hline
Masculin & der & dem & einem  \\
Feminin & die &  der & einer \\
Neutrum & das & dem & einem \\
Plural & die & den & -n \\
\hline
\end{tabular}

Para el plural, agregar \textbf{-n} al final, si la palabra no tiene ya una n al final.

Pronombres:

\begin{tabular}{|c | c | c |}
\hline
\textbf{Género} & \textbf{Akkusativ} & \textbf{Dativ} \\
\hline
ich & mich & mir \\
du & dich & dir \\
er & ihn & ihm \\
sie & sie & ihr \\
es & es & ihm \\
wir & uns & uns \\
ihr & euch & euch \\
Sie/sie & Sie/sie & Ihnen/ihnen \\
\hline
\end{tabular}

Preposiciones con las que se usa:
\begin{multicols}{3}
\begin{myitemize}
\item aus: de
\item außer: excepto
\item bei: en
\item mit: con
\item nach: hacia
\item seit: desde (tiempo)
\item von: desde (lugar)
\item zu: hacia
\end{myitemize}
\end{multicols}

Verbos con los que se usa:
\begin{multicols}{2}
\begin{myitemize}
\item antworten: responder
\item befehlen: pedir
\item danken: agradecer
\item erzählen: narrar
\item fehlen: carecer
\item folgen: seguir
\item gefallen: gustar
\item gehören: pertenecer
\item glauben: creer
\item gratulieren: felicitar
\item helfen: ayudar
\item liegen: estar
\item passen: quedar
\item sagen: decir
\item schmecken: probar
\item schreiben: escribir
\item sitzen: sentar
\item stehen: estar de pie
\item vertrauen: confiar
\item zuhören: escuchar
\end{myitemize}
\end{multicols}

\subsection{Genitivo - Genitiv}
Próximamente...

\subsection{Acusativo o Dativo}
Hay algunas preposiciones que se pueden usar con ambos casos:
\begin{myitemize}
\item an: contra
\item auf: sobre
\item hinter: detrás
\item in: en
\item neben: al lado de
\item über: sobre
\item unter: bajo
\item vor: delante
\item zwischen: entre
\end{myitemize}

Si la preposición responde a la pregunta \textit{dónde} (wo), se usa el dativo. Wo bleibst du? In die Schule. Se utiliza también para todos los verbos de posición:
\begin{myitemize}
\item bleiben: permanecer
\item hängen: estar colgado
\item liegen: estar acostado
\item schlafen: dormir
\item sein: estar
\item sitzen: estar sentado
\item stehen: estar de pie
\item warten: esperar
\item wohnen: vivir
\end{myitemize}

Si la preposición responde a la pregunta \textit{hacia dónde} (wohin), se usa el acusativo. Wohin gehst du? In die Schule. Se usa también para los verbos de movimiento:
\begin{myitemize}
\item fahren: conducir
\item fliegen: volar
\item gehen: ir
\item hängen: colgar
\item kommen: venir
\item legen: acostarse
\item setzen: sentarse
\item stellen: posar
\end{myitemize}

\subsection{Otras declinaciones}
\begin{tabular}{|c | c | c | c |}
\hline
\textbf{Género} & \textbf{Nominativ} & \textbf{Akkusativ} & \textbf{Dativ}\\
\hline
Maskulin & welcher & welchen & welchem  \\
Feminin  & welche  &  welche & welcher \\
Neutral  & welches & welches & welchem \\
Plural   & welche  & welche  & welchen \\
\hline
\end{tabular}

\begin{tabular}{|c | c | c | c |}
\hline
\textbf{Género} & \textbf{Nominativ} & \textbf{Akkusativ} & \textbf{Dativ}\\
\hline
Maskulin & dieser & diesen & diesem  \\
Feminin  & diese  & diese  & dieser \\
Neutral  & dieses & dieses & diesem \\
Plural   & diese  & diese  & diesen \\
\hline
\end{tabular}

\section{Orden de palabras en una oración}
Este es el orden de palabras dentro de una oración o proposición:
\begin{myitemize}
\item S: sujeto
\item V: verbo
\item D: dativo
\item T: modificador temporal
\item C: modificador causal
\item M: modificador modal
\item A: acusativo
\item L: modificador local
\item V2: auxiliar del verbo
\end{myitemize}

\section{Números}
\subsection{Números ordinales}
Los números ordinales son utilizados para indicar un orden. También se usan para las fechas: am ersten Januar = 1. Januar = Primero de Enero. Para los números entre el 1 y el 19, se usa el sufijo \textbf{-ten}. Para los números mayores o iguales a 20, se usa el sufijo \textbf{-sten}. Hay también algunas excepciones en la forma de escribir.
\begin{multicols}{2}
\begin{myitemize}
\item 1: ersten
\item 2: zweiten
\item 3: dritten
\item 4: vierten
\item 5: fünften
\item 6: sechsten
\item 7: siebten
\item 8: achten
\item 9: neunten
\item 10: zehnten
\item 11: elften
\item ...
\item 20: zwanzigsten
\item 21: einundzwanzigsten
\item 22: zweiwundzwanzigsten
\item ...
\end{myitemize}
\end{multicols}

\subsection{Fracciones}
\begin{multicols}{2}
\begin{myitemize}
\item 1/2: ein halb
\item 1/3: ein drittel
\item 1/4: ein viertel
\item 1/20: ein zwanzigstel
\end{myitemize}
\end{multicols}
Nota: la palabra halb cumple la función de adjetivo y por lo tanto debe declinarse: \textit{ein halbe Liter Wasser}.

\section{Comparativos y Superlativos}
\begin{tabular}{| c | c | c | c | c |}
\hline
\textbf{Adjetivo} & \textbf{Regla} & \textbf{Ejemplo} & \textbf{Komparativ} & \textbf{Superlativ} \\
\hline
General & -er, sten & klein & kleiner & am kleinsten \\
-t, -d, -s, -z & -er, -esten & laut & lauter & am lautesten \\
o - ö, a - ä, u - ü & \"{ }-er, \"{ }-sten & groß & größer & am größten \\
Irregular & - & teuer & teurer & am teuersten \\
Irregular & - & viel & mehr & am meisten \\
Irregular & - & gern & lieber & am liebsten \\
Irregular & - & gut & besser & am besten \\
\hline
\end{tabular}

Ejemplos: 
\begin{myitemize}
\item Der Zug ist \textbf{schnellner} als das Fahrrad. Das Flugzeug ist \textbf{am schnellsten}.
\item Dr. House ist \textbf{der beste} Artz.
\end{myitemize}

\section{Preposiciones y contracciones}
\begin{tabular}{| c | c | c |}
\hline
\textbf{Preposición} & \textbf{Partícula} & \textbf{Resultado} \\
\hline
an & das & ans \\
an & dem & am \\
auf & das & aufs \\
bei & dem & beim \\
in & das & ins \\
in & dem & im \\
zu & dem & zum \\
zu & der & zur \\
\hline
\end{tabular}

\section{Adjetivos sustantivados}
Los adjetivos sustantivados son sustantivos formados a partir de un adjetivo: gut $\rightarrow$ das Gute.

Siempre van acompañados de un artículo y siguen las reglas de declinación de los adjetivos. En la mayoría de los casos, se declinan añadiendo la terminación \textbf{-en}.

\begin{tabular}{|c | c | c | c |}
\hline
\textbf{Género} & \textbf{Nominativ} & \textbf{Akkusativ} & \textbf{Dativ}\\
\hline
Maskulin & der Neue  & den Neuen & dem Neuen \\
Feminin  & die Neue  & die Neue  & der Neuen \\
Neutral  & das Neue  & das Neue  & dem Neuen \\
Plural   & die Neuen & die Neuen & den Neuen \\
\hline
\end{tabular}

\begin{tabular}{|c | c | c | c |}
\hline
\textbf{Género} & \textbf{Nominativ} & \textbf{Akkusativ} & \textbf{Dativ}\\
\hline
Maskulin & ein Neuer  &  einen Neuen & einem Neuen \\
Feminin  & eine Neue  &  eine Neue   & einer Neuen \\
Neutral  & ein Neues  &  ein Neues   & einem Neuen \\
Plural   & eine Neuen &  einen Neuen & einen Neuen \\
\hline
\end{tabular}

\section{N-deklination}

Algunos nombres masculinos deben ser declinados con esta regla al agregarles \textbf{-n} o \textbf{-en}.
\begin{myitemize}
\item \textbf{-e}: der Junge, der Kunde
\item \textbf{-ent}: der Student
\item \textbf{-ant}: der Praktikant
\item \textbf{-ist}: der Journalist
\item Procendentes del griego: der Soldat, der Architekt
\item Otros: Mensch, Nachbar, Herr
\end{myitemize}

\begin{tabular}{|c | c | c |}
\hline
\textbf{Nominativ} & \textbf{Akkusativ} & \textbf{Dativ}\\
\hline
der Name & den Namen & dem Namen \\
der Journalist & den Journalisten & dem Journalisten \\
der Dozent & den Dozenten & dem Dozenten \\
\hline
\end{tabular}

\section{Oraciones complejas}
\subsection{Cláusula subordinada - Nebensatz}
Una oración compleja (Satzgefüge) está formada por una cláusula principal (Hauptsatz) y una cláusula subordinada (Nebensatz). Para formarla, se usan conjuciones, por ejemplo \textbf{weil} después de la coma, seguida de la justificación, pero con el verbo conjugado al final.

Ejemplos: 
\begin{myitemize}
\item Weil/Porque: Ich backe einen Kuchen, \textbf{weil} mein Freund Geburstag \textit{hat}. Yo cocino un pastel \textbf{porque} mi amigo \textit{está} de cumpleaños.
\item Dass/Que: Mein Bruder sagt, \textbf{dass} das Buch zu teuer \textit{ist}. Mi hermano dice \textbf{que} el libro \textit{es} muy caro.
\item Ob/Si: Ich frage deinen Freund, \textbf{ob} er einen Garten \textit{hat}. Le pregunto a tu amigo \textbf{si} él \textit{tiene} un jardín.
\item Wenn/cuando: Ich esse einen Hamburger, wenn ich einen Kater habe. Como una hamburguesa cuando tengo una resaca.
\item obwohl: Es hat sich bis Heute nichts geändert, obwohl ich Ihnen des mehrfach erklärt habe: Nada ha cambiado a día de hoy, aunque te lo he explicado varias veces 
\end{myitemize}

Con algunas conjunciones, la cláusula subordinada también puede ir al inicio de la oración:
\begin{myitemize}
\item obwohl: Obwohl es stark regnet, gehen die Kinder in den Park: Aunque llueve mucho, los niños van al parque.
\end{myitemize}

\subsection{Cláusulas relativas - Relativsätze}
Las cláusulas relativas sirven para hacer referencia a algo que se mencionó antes sin repetirlo. Para formarlas, se debe separar las cláusulas con una coma, poner la preposición (si necesario), el pronombre (ver tabla) que hace referencia al elemento repetido, y ubicar el verbo al final. El pronombre a utilizar depende de la función que tiene el elemento repetido en la cláusula principal.

Ejemplo: Das sind die Freunde. Ich verbringe mit den Freunden viel Zeit. 

En este caso, \textit{Freunde} está repetido y está en dativo. La oración reformulada con cláusula relativa es: Das sin die Freunde, mit denen ich viel Zeit verbringe.

Otro ejemplo en el cual se ve que la cláusula relativa está en el medio de la oración: Viele Leute, die auf dem Zug warteten, standen am Bahnsteig.


La siguiente tabla muestra los pronombres a utilizar como conectores, según el caso.

\begin{tabular}{|c | c | c | c |}
\hline
\textbf{Género} & \textbf{Nominativ} & \textbf{Akkusativ} & \textbf{Dativ}\\
\hline
Maskulin & der/welcher & den/welchen & dem/welchem  \\
Feminin  & die/welche  & die/welche  & der/welcher \\
Neutral  & das/welches & das/welches & dem/welchem \\
Plural   & die/welche  & die/welche  & denen/welchen \\
\hline
\end{tabular}

\subsection{Dos cláusulas principales}
En este caso, se trata de dos cláusulas principales, conectadas por algún conector. Como se trata de cláusulas principales, el verbo no va al final sino al principio, pero hay una inversión.

Ejemplos:
\begin{myitemize}
\item trotzdem: Es regnet sehr stark, trotzdem wollen die Kinder im Park spielen: Llueve muy fuerte, sin embargo los niños quieren. jugar en el parque.
\end{myitemize}

\subsection{Doch}
\textbf{Doch} tiene varios usos.

Para contrarrestar un negativo:
Has du kein Geld? Doch!: No tienes dinero? Sí, si tengo.

Para decir ``finalmente'' o ``en realidad'':
Ich gehe doch nicht ins Kino: Finalmente no voy al cine.
Der Film gefällt mir doch: En realidad sí me gustó la película.

Para decir ``pero'':
Ich bin müde, doch ich muss diesen Post zuende lesen. Estoy cansado, pero debo terminar de leer este post.
